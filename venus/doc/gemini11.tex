\documentstyle[12pt]{article}
\pagestyle {empty}
\setlength {\parindent}{0cm}
\addtolength {\leftmargin}{3cm}
\addtolength {\textwidth}{-3cm}

\begin{document}
\sf
\centerline{\Large{\bf Gemini 1.1}}
\vspace*{1cm}
Zwei Monate nach Ver"offentlichung der ersten Fassung von
{\em Gemini}, dem Shareware-Desktop, liegt nun die erste
"uberarbeitete Version mit der Versionsnummer 1.1 vor.
Wichtigste Neuerungen: die Shell kennt jetzt die Funktionen
\verb?more? und \verb?print? zum Anzeigen und Ausdrucken von
Dateien. Diese Funktionen k"onnen auch vom Desktop per Mausklick
aufgerufen werden. Ferner wird der neue Atari-Standard \verb?ARGV?
zur "Ubergabe erweiterter Kommandozeilen benutzt. Der Desktop selbst
ist jetzt standardm"a"sig deutschsprachig,
hat noch mehr Standardicons als bislang und verf�gt �ber eine nochmals
erweiterte Kopierfunktion f�r solche Icons, die bereits auf dem
Desktophintergrund abgelegt sind.

Alle registrierten Anwender erhalten den Update auf die neue Version
umsonst. Die neue Version ist wieder "uber alle gutsortierten
Mailboxen oder gegen Einsendung einer formatierten Diskette mit
frankiertem R"uckumschlag direkt bei den Autoren erh"altlich. Die
Shareware-Geb"uhr betr"agt noch immer 50 Mark.

\vspace{1cm}
{\obeylines
\centerline{Weitere Information bei}

Stefan Eissing \hfill Gereon Steffens 
Dorfbauernschaft 7 \hfill Elsterweg 8
4419 Laer-Holthausen \hfill 5000 K"oln 90
Mailbox: MAUS M"unster \hfill Mailbox: MAUS Bonn
(Telefon 0251/80386) \hfill (Telefon 0228/254020)
}

\end {document}